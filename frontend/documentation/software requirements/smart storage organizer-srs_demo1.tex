\documentclass[hidelinks, 12pt, a4paper]{article}
\usepackage[utf8]{inputenc}
\usepackage{float}
\usepackage[english]{babel}
\usepackage{hyperref}
\usepackage{mwe}
\usepackage{ragged2e}
\usepackage[letterpaper,top=2cm,bottom=2cm,left=3cm,right=3cm,marginparwidth=1.75cm]{geometry}
\usepackage{graphicx, wrapfig}
\usepackage{pgf}
\usepackage{pgfpages}
\usepackage{multirow}
\usepackage{xcolor}

\title{Smart Storage Organizer}
\author{Mosa Letswalo, Victor Zhou, Tshegofatso Mapheto,Siyabonga Mbuyisa, Ezekiel Makau }

\pgfpagesdeclarelayout{boxed}
{
  \edef\pgfpageoptionborder{0pt}
}
{
  \pgfpagesphysicalpageoptions
  {%
    logical pages=1,%
  }
  \pgfpageslogicalpageoptions{1}
  {
    border code=\pgfsetlinewidth{2pt}\pgfstroke,%
    border shrink=\pgfpageoptionborder,%
    resized width=.95\pgfphysicalwidth,%
    resized height=.95\pgfphysicalheight,%
    center=\pgfpoint{.5\pgfphysicalwidth}{.5\pgfphysicalheight}%
  }%
}

\pgfpagesuselayout{boxed}

\begin{document}

\begin{titlepage}
\begin{figure}
    \centering\includegraphics[width=0.35\textwidth]{tuks_logo.png}
\end{figure}
\vspace{5mm}

\begin{Large}
 \begin{center}
	\textbf{Department of Computer Science}\\
	\vspace{8mm}
    \textbf{Software Requirements and Design Specifications}\\
	\vspace{8mm}
	{\huge{\bf Smart Storage Organizer}}\\
	\textbf{Client: DVT/Shamier Chandley}\\
	\vspace{8mm}
	{\huge{\bf Team: Syntax Savants}}

	\textbf{Team Members: }

    \begin{table}[h]
        \centering
        \begin{tabular}{|c|c|c|c|}

            \hline
            Name        &     Surname   &   Student Number  \\ \hline
            *Siyabonga     &     Mbuyisa     &  u20491621  \\ \hline
            Tshegofatso      &    Mapheto   &  u20735929        \\ \hline
            Mosa        &   Letswalo    &   u14168970      \\ \hline
            Victor     &    Zhou       &   u21525936       \\\hline
            Ezekiel    &   Makau   &   u20702338       \\ \hline
        \end{tabular}
    \end{table}
    \textit{* - indicates team leader}
\end{center}
\end{Large}
\end{titlepage}

\setcounter{tocdepth}{4}
\setcounter{secnumdepth}{4}
\tableofcontents
\newpage

\section{Introduction}
% Explain the vision and objectives. State the business need for the application and summarise the scope of the
% project.
\subsection{Vision and Objectives}
Traditional storage organization methods often rely on manual labelling and searching, leading to wasted time and frustration.
\newline
This project focuses on creating an application that will help users efficiently organize and easily locate their belongings. It aims to solve the problems associated with traditional storage methods, such as manual labelling and searching, which often result in wasted time and frustration
\newline
\newline
The objective of this project is to design and implement a system that simplifies item labelling and identification using QR codes and color coding. The key features of the system include:
\begin{itemize}
    \item Simplify item labeling and identification using QR codes and color coding.
    \item Enhance searchability through text and voice search functionalities.
    \item Offer potential for enhanced search accuracy using AI integration.
\end{itemize}
% got the description from the project proposal

\subsection{Project Owner}
Shamier Chandley | Dynamic Visual Technologies (dvt).

\pagebreak

\section{User Characteristics}
% List all intended users and explain for what purpose each of them would use the system.
All of the below-mentioned users are assumed to have knowledge of operating a computer and using a smartphone. They should also have access to the internet.

\subsection{Employees}
\begin{itemize}
    % \item A person that works for dvt.
    \item A person that will regularly store, organize and label items.
\end{itemize}
\subsection{dvt Admins}
\begin{itemize}
    \item A person that we will manage the application.
    \item A person that will routinely manage both the dvt Employees.
\end{itemize}

\pagebreak

\section{Functional Requirements}

\subsection{Core Requirements}
\begin{itemize}
    % CORE:
    \item \textbf {R1}: {User Management: Users can register, log in, and manage their profiles}

    % CORE:
    \item \textbf{R2}: {Item Management: Users can add, edit, and delete items within the application.}

    % CORE:
    \item \textbf{R3}: {QR Code Integration: Users can generate and associate unique QR codes with individual items or a group of similar items. }

    % CORE:
    \item \textbf{R4}: {Color Coding: Users can assign color codes to items for categorization and filtering.}

    % CORE:
    \item \textbf{R5}: {Text Search: Users can search for items using keywords and filter results based on various criteria.}

    % CORE:
    \item \textbf{R6}: {Voice Search: Users can use voice commands to search for items.}

    % CORE:
    \item \textbf{R7}: {Push Notifications: Users can receive optional real-time notifications about application updates or item-related alerts.}

    % CORE:
    \item \textbf{R8}: {Data Persistence: Application data, including item details and user information, is securely stored in a persistent database.}

    % CORE:
    \item \textbf{R9}: {Advanced search with AI integration: Implement advanced search filters based on additional item details and characteristics.}

    % CORE:
    \item \textbf{R10}: {Image capture: Allow users to capture images of items for better identification.}

    % CORE:
    \item \textbf{R11}: {AI Integration: Utilize open-source APIs like ChatGPT to enhance search accuracy and understand user intent.}

    \item \textbf{R12}: {Data Security: User data must be protected through robust security measures.}
\end{itemize}

\subsection{Nice to have Requirements}
\begin{itemize}
    % Nice-to-have:
    \item \textbf{R13}: {Offline Functionality: The application should function to some extent even without an internet connection (optional).}

    % Nice-to-have:
    \item \textbf{R14}: {Barcode scanning: Allow users to scan existing barcodes.}

    % Nice-to-have: (Really not though if you ask clients)
    \item \textbf{R15}: {Location tracking: Enable users to mark the general location of stored items within the application.}

    % Nice-to-have: (Really not though if you ask clients)
    \item \textbf{R16}: {Multi-language support: Support multiple languages for broader accessibility.}

    % Nice-to-have: (Really not though if you ask clients)
    \item \textbf{R17}: {Sharing functionality: Enable users to share item information with others.}

    % Nice-to-have: (Really not though if you ask clients)
    \item \textbf{R18}: {User roles and permissions: Define different user roles with varying access levels.}

    % Nice-to-have: (Really not though if you ask clients)
    \item \textbf{R19}: {Cloud storage integration: Utilize cloud storage services for backup and disaster
    recovery.}
\end{itemize}
\section{User Stories}
\begin{itemize}
    \item \textbf{R1: Employee Management}
    \begin{itemize}
        \item \textbf{U1.1 Employee registration}
        \begin{itemize}
            \item As an employee
            \item I want to register my details so that I can use the app.
            \item I want to be able to manage my profile and be able to update and reset my password.
        \end{itemize}
        \item \textbf{U1.2 Employee login}
        \begin{itemize}
            \item As an employee
            \item I want to be able to login to the system in order to manage items in the storage.
        \end{itemize}
        \item \textbf{U1.3 Employee profile}
        \begin{itemize}
            \item As an employee
            \item I want to manage my profile. So that I can be able to update my password and my personal details.
        \end{itemize}
    \end{itemize}

    \item \textbf{R2: Item management}
    \begin{itemize}
        \item \textbf{U2.1 Add item}
        \begin{itemize}
            \item As an employee
            \item I want to add and save the items I'd like to store. So that I can be able to access them at a later stage.

        \end{itemize}
        \item \textbf{U2.2 Delete item}
        \begin{itemize}
            \item I want to delete items that I no longer need in the storage. So that I can create space for new items.
        \end{itemize}
        \item \textbf{U2.3 edit item}
        \begin{itemize}
            \item As an employee
            \item I want to edit my existing items. So that I can be able to update any changes on the existing items.
        \end{itemize}
    \end{itemize}

    \item \textbf{R3: QR code}
    \begin{itemize}
        \item As an employee
    \end{itemize}

    \item \textbf{R4: Color Coding }
    \begin{itemize}
        \item As an employee
    \end{itemize}

    \item \textbf{R5: Text search}
    \begin{itemize}
        \item As an employee
    \end{itemize}

    \item \textbf{R6: Voice search}
    \begin{itemize}
        \item As an employee
    \end{itemize}

    \item \textbf{R7: Push notification}
    \begin{itemize}
        \item As an employee
    \end{itemize}

    \item \textbf{R8: Data persistence}
    \begin{itemize}
        \item As an employee
    \end{itemize}

    \item \textbf{R9: Advanced search}
    \begin{itemize}
        \item As an employee
    \end{itemize}

    \item \textbf{R10: Image capture}
    \begin{itemize}
        \item As an employee
    \end{itemize}

    \item \textbf{R11: AI integration}
    \begin{itemize}
        \item As an employee
    \end{itemize}
% Should this be under quality requirements...?
    \item \textbf{R12: Data security}
    \begin{itemize}
        \item As an employee
    \end{itemize}

% FOR THE NICE TO HAVE REQUIREMENTS WHICH ONES ARE WE DOING ? Can we choose Wow factor from here ??
    \item \textbf{R13: Offline functionality}
    \begin{itemize}
        \item As an employee
    \end{itemize}

    \item \textbf{R14: Barcode scanning}
    \begin{itemize}
        \item As an employee
    \end{itemize}

    \item \textbf{R15: Location tracking}
    \begin{itemize}
        \item As an employee
    \end{itemize}

    \item \textbf{R16: Multi-language support}
    \begin{itemize}
        \item As an employee
    \end{itemize}

    \item \textbf{R17: Sharing functionality}
    \begin{itemize}
        \item As an employee
    \end{itemize}

    \item \textbf{R18: User roles and permissions}
    \begin{itemize}
        \item As an employee
    \end{itemize}

    \item \textbf{R19: Cloud storage integration}
    \begin{itemize}
        \item As an employee
    \end{itemize}
    \pagebreak
